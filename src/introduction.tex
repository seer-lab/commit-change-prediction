\chapter{Introduction}
\label{chap:introduction}

% intro

% Garbage.
This thesis generally covers topics relating to effort estimation and planning of project development. The development of a software project can vary greatly based on the scope of the project. Larger scale projects that have a complex task or set of tasks to accomplish often require a long period of time with a committed team of developers. Even once a project completely performs a task further development is needed to maintain the project for the remainder of its life.

For the development of software in a commercial setting the ability for managers to identify the cost of a project is essential for effective business decisions. Effort estimation is one possible avenue for project managers to leverage to identify the complexity of a project and associated cost of that project.

The ability of developers or managers to extract more information from a project is essential to helping them make more informed decisions about the development of the project. For example if a developer can identify a location within a project that is very likely to receive changes in future development then the development may be more inclined carefully consider the types of changes necessary to make.

% Thesis statement
We propose a tool that assists in managing the development of a software project by predicting which changes will occur. This work explores leveraging change prediction of the source code using the change history to assist in the development of large scale projects.

% Contributions
Our contributions are in mining of \gls{oss}, visualization of a project's change history, machine learning change prediction, data collect which can be used and extended.

% ---Explain

Mining of open source projects has been widely used to help research into various software topics relating to project development and quality assurance. This research is vital to improving the development process of software projects. By improving the development of software projects more may succeed in accomplishing their outlined goal. The project development process will take time to complete. The time it takes for the project to be completed relies on numerous factors including project scope, man power, experience. Over the course of the project development changes will be made to project. Changes can be made to almost any part of the project including design, number of developers and type of developers. These changes will in most cases have a measurable impact on the project (or at least they are intended to). In case of adding more developers the intended result may be to increase project capabilities within a shorter span of time than previously. Even with an intended result, the actual result may differ and should be measured to determine the effectiveness of a given change.

The developers of the project must therefore manage changes made to the project to ensure that the changes that are made result in the expected outcome. Keeping track of every change to a project can be difficult because of external changes which are beyond the control of the developers. However for the majority of the changes within the project they can be kept track by using a \gls{vcs}. With proper use of a \gls{vcs} the important changes made to the project will be stored. This can help keep previous releases of the software available or even help resolve a bug that was introduced in a recent change. With numerous developers a \gls{vcs} can also help improve how these developers interact and share the changes that they are making.

The impact of changes can be measured and provide insights into how the project changes. However first the data must be collected and then processed into a usable form.

The source code level changes with a project map directly to functionality changes. Whether the such a change is new, fixed or removed functionality. Simply observing source code line changes can encounter a large amount of noise within which can make tracking the desired changes more difficult.

One such change that can be made to the software project would be source code changes. These changes are very fine grain since they will account for almost all functionality changes with the project. 

however the actual processing of the data will vary depending on the application. Analysis of change data requires extracting data for a large set of data.

Visualization of the data collected allows for a more accessible look at the data to provide potential insights.

The change prediction process leverages machine learning techniques to train based on the data collected through mining GitHub.

% Organization
% TODO talk about how the rest of the paper is organized.

The rest of the this thesis is organized into 4 more chapters. \hyperref[chap:related_works]{Literature Review}, \hyperref[chap:approach]{Approach}, \hyperref[chap:experiments]{Experiments} and finally the \hyperref[chap:conclusions]{Conclusion}. In chapter \ref{chap:related_works} more details are given related to the foundation of this work. Primarily this will cover the data that is collected for the analysis. The following chapter \ref{chap:approach} discusses the change prediction approach from how the data is collected and stored to what methods are used for to predict change within the project. Chapter \ref{chap:experiments} reports the experiments conducted and their results. Finally the paper the conclusion summarizes the results and contributions and proposes future work to build of the thesis.