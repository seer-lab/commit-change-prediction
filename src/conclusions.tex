\chapter{Conclusions}
\label{chap:conclusions}


We proposed a method that leverages the commit history to predict future changes within the project. The changes that are predicted are in the short term of 5 commits. The approach was then tested on five different \gls{oss} projects developed in Java. The tests investigated the different factors that impacted the performance of the approach. The results of the tests show that while the \gls{swr} had a strong impact on the performance, the projects themselves often had internal factors which caused differences in performance.

The contributions of this work are:
\begin{enumerate}
\item Providing an approach that with some success can predict future changes within a project using the commit data. Both \gls{svm} and \gls{rf} are viable for some projects.
\item Determined which factors more strongly influence the performance of the predictions. Out of the three factors investigated, the \gls{swr} proved to have the greatest impact for both \gls{svm} and \gls{rf}.
\end{enumerate}

Future work includes investigating the projects and they differ in the way they change. The two projects that tended to do will together were acra and dagger which also shared similar project characteristics. Finally a more extensive look at the other factors that were involved in the approach to determine their impact of the approach.

% TODO:
% - summary
% - contribution
%     - restated, in more depth
% - limitations
%     - Based around using GitHub data set
%     - as well as limited data from git and open source
% - future work
% - conclusion section
%     - summarize what we've done, what can we really learn about prediction and our our approach
%     - how does this compare to results we've archived version other methods


% TODO 
% either talk about this here, or in the dicussions section of the rf and svm experiments.
% - limited range of swr, wasn't exhaustive.
% - limited set of factors, wasn't exhaustive.